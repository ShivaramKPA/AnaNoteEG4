\chapter{Conclusion}
\label{cha:Conclusion}

The multiplicity, $\cos\phi_h$ moment, and $\cos 2\phi_h$ moment of the unpolarized SIDIS cross-section have been measured for both charged pion channels in a fully differential way with good statistics over a wide kinematic range ($0.1 < x < 0.6$, $1.0 < Q^2 < 4.7\ \text{GeV}^2$, $0.0 < z < 0.9$, $0.0 < P_{h\perp}^2 < 1.0\ \text{GeV}^2$, $-180^\circ < \phi_h < 180^\circ$).
The $\cos\phi_h$ and $\cos 2\phi_h$ modulations show a clear dependence on flavor which hints at a non-zero Boer-Mulders effect~\cite{Gamberg08}\cite{Barone08}\cite{Zhang08}\cite{Barone10} and could give insights into the quark orbital angular momentum ($L_q$) contribution of the proton spin; but more intensive theoretical comparisons, which are currently in progress, are needed first.
Results show a reasonable agreement with a previous CLAS measurement in overlapping kinematics.

The study of nucleon structure via the accessing of TMDs is a major thrust of up-and-coming nuclear physics projects such as CLAS12~\cite{Burkert08} and the proposed Electron-Ion Collider (EIC)~\cite{Boer11}.
A CLAS Collaboration proposal to continue studies of the Boer-Mulders function at higher energies at CLAS12 has been given the highest priority rating by PAC 30~\cite{Avakian06}.
Many of the challenges associated with precision measurements of this type have been addressed here.
Furthermore, the two experiments (this one and the future one at CLAS12) will have some overlapping kinematics and some unique kinematics, and therefore this analysis will be valuable both in its own right and for comparison purposes.
In conclusion, the work described in this dissertation has improved our understanding of this field and provided a solid foundation for future advancements.

