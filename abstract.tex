\abstract{
%The main goal of the EG4 experiment is to measure the spin structure function \gones of the proton and the neutron, its first moment Γ1 and then the GDH integral in the very low momentum transfer regime where it will provide a test for the chiral-perturbation theory as the low-energy QCD approximation which makes stringent predictions in this regime. Here, I will give a brief overview of the experiment. I will also give the motivation, procedure and the results of the EC-timing calibration work and raster-correction work that I was involved in. A brief general overview of the future work is also given.\textcolor{red}{Two paragraphs not allowed in an abstract}
Double polarization %Longitudinal polarized 
cross section differences ($\Delta \sigma_{\parallel}$) 
%\textcolor{red}{Or count differences, instead?}
for proton and deuteron targets have been measured in the EG4 experiment using the CLAS detector at Jefferson Lab. Longitudinally polarized electron beams at relatively low energies of 1.056, 1.337, 1.989, 2.256 and 3.0 GeV from the CEBAF accelerator were scattered off longitudinally polarized NH$_3$ and ND$_3$ targets. % using DNP technique for polarization
%Resulting forward scattered  SEKc
Scattered electrons were recorded at very low scattering angles (down to $\theta = 6^o$) with the help of a new dedicated Cherenkov counter and a special magnetic field setting of the CLAS detector
% \textcolor{red}{I mean "the outbending torus field" -should I be explicit?}
in order to measure the cross section differences in the resonance region ($1.08$ ~GeV$< W <2.0$ ~GeV) 
at very low momentum transfers (\qsqs for the deuteron was as low as 0.02 GeV$^2$). These measurements on the deuteron %\textcolor{red}{Now I am focussing on deuteron alone}
were used to extract the deuteron's spin structure function \gones as well as the product \afones of the virtual photon asymmetry $A_1$ and the unpolarized structure function $F_1$. %\textcolor{red}{Dr. Dodge (GED): Must define in words what A1F1 is?}
These extracted quantities, in turn, were used to evaluate three important integrals for the deuteron - the first moment ($\Gamma_1$) of \gone, the extended Gerasimov-Drell-Hearn (GDH) integral ($\bar{I}_{TT}$),  and the generalized forward spin polarizability ($\gamma_0$). These measurements extend and improve the world deuteron data on \gones to the previously  unmeasured %/unexplored 
low \qsqs region. %This 
The data, in combination with the corresponding proton data from the same experiment, will be valuable to extract \gones on the neutron in the same kinematics. %These data
They will shed more light on the nucleon spin structure in the region of quark-confinement as well in the transition region between hadronic and partonic degrees of freedom. In addition, the three integrals evaluated from the measured data are compared to %with Dr. Gail Dodge (to/with)
 predictions from different %calculations %versions/approximations of 
Chiral Perturbation Theory (\chipt) calculations and phenomenological models. %These data will also serve as benchmarks for similar future \chipt and phenomenological calculations. These measurements will also allow extrapolations
Extrapolations of the integrals (especially the GDH sum and the polarizability) to the real photon point (\qsq=0) %enabling 
enable us to test the validity of the predictions for %on 
their real photon counterparts. % and helping us to undestand the properties of the few-body nuclei.
%\textcolor{red}{GED: Say something about the result. What did we find?}
The new results have extended and improved the very low \qsqs data on \gones and the corresponding results on moments compare very well with the latest \chipts and phenomenological calculations (especially near the photon point). %\textcolor{red}{Last sentence new.}
}
