\clearpage
%
\chapter{Conclusions}
\label{cha:conclusions}




% Started this on a hurry on 11/27/13: 
% I guided by myself by the following info: 
% SEK email: a final chapter "Conclusions" which contains at least a couple of pages, summarizing the experiment and its results.
% SEK comment on earlier draft (pg 192): A whole chapter is missing: Conclusion. Summarize the main features of our data (low Q^2) and the main results we find. outlook on similar proton results. Hope for unified description of SSFs over all kinematic regions.
% # of pages on other theses: Nevzat: 1.5, Slifer: 2.3, Sulkosky: 1.3, Mike Mayer: 1.3
\hspace{0.5cm}
% \newline \newline

%\section{Introduction}
%\textcolor{red}{Colored texts are either temporary}.

%Features of EG4 data
The very low momentum transfer (\qsq) data from the EG4 experiment have been analyzed for the helicity 
dependent inclusive cross section (difference) for the scattering of longitudinally polarized 
electrons off longitudinally polarized deuterons (from DNP polarized \nd3 target, using two beam energies 1.3 and 2.0 GeV). %\nh3 and \nd3 targets respectively).
 The analyzed data has the kinematic coverage of $~$ ($0.02$ GeV$^2 ~ \lt ~Q^2 ~ \lt ~ 0.7 ~$ GeV$^2$) and ($1.08$ GeV$ ~ \lt ~W ~ \lt ~ 2.0 ~$ GeV). Although past measurements from EG1b go as low as $0.05$ GeV$^2$ in \qsq, the new measurements have higher precision (due to higher statistics and better detection efficiency) in the overlapping region (particularly evident below \qsq = 0.2 GeV$^2$) in addition to new high precision data in the previously unmeasured region below \qsqs = 0.5 GeV$^2$. % of ($0.02$ GeV$^2 ~ \lt ~Q^2 ~ \lt ~ 0.05 ~$ GeV$^2$). 
%From my defense prep
%Measurement of spin structure function g1(x,Q2) & its moments at low Q2 sheds light on the internal nucleon structure & the transition from partonic to hadronic descriptions of strong interaction.
%Polarized electron scattering data on polarized proton and deuterons covering resonance region down to very low Q2 has been collected by EG4 experiment in JLab Hall-B.
%First preliminary results for g1(x,Q2) for deuteron has been obtained by comparing data to corresponding simulation and they have been used to evaluate some integrals/moments to test Chiral Perturbation Theory and phenomenological predictions.
%Very preliminary data on proton g1(x,Q2) also becoming available.


% %Main results
%Extraction of g1, A1F1 and integrals
The new deuteron data were used to extract the deuteron's spin structure function \gones by 
comparing the experimental data with simulated data produced by using a 
%more or less %SEK: say realistic, don't insult my model :) 
realistic cross section model for the deuteron under similar kinematic conditions. %\goness as well as the product \afones of the virtual photon asymmetry \aones and the unpolarized structure function \fones. The results of \gones and \afone, in turn, were used to evaluate three important moments of the structure functions.
%
%Following is copied from the first page of Results chapter.
The newly extracted data pushes the lower limit on \qsqs in the resonance region with reduced 
systematic and statistical uncertainties that will contribute greatly to the world data set. 
It is observed that the data from two beam energies give results that are in good agreement 
wherever they overalp. The low \qsqs results clearly show resonance structure in the region 
$W \le 2.0$ which smooths out as \qsqs becomes larger. In particular, the $\Delta$-resonance 
shows a strongly and consistently negative signal at all \qsq, but the second resonance region 
(around W=1.5 GeV) shows a rather %unexpected 
rapid transition of \gones (or polarized cross section) from 
strongly negative values at low \qsqs to clearly positive values at high \qsq. 
%The latter observation %
This observation in the second resonance region 
is not well described by the model because the model is not constrained in the region due to the lack 
of experimental data up to now and indicates that the spin-flip helicity amplitude $A^T_{\frac{3}{2}}$ dominates 
the cross section at low \qsqs while the non-flip amplitude $A^T_{\frac{1}{2}}$ becomes stronger 
at higher \qsq.


%The new deuteron data were used to extract the deuteron's spin structure function \gones as well as the product \afones of the virtual photon asymmetry \aones and the unpolarized structure function \fones. The results of \gones and \afone, in turn, were used to evaluate three important moments of the structure functions.
The product \afones of the virtual photon asymmetry \aones and the unpolarized structure function 
\fones was also extracted from the same %deuteron target 
data %set by using the same method. 
and method.
The extracted results on \gones and \afones were then used to evaluate %the following 
three important moments - the first moment $\Gamma^d_1$ of \gone, the generalized GDH integral $\bar{I}^d_{TT}$ and the generalized forward spin polarizability $\gamma^d_0$ - in each of the \qsqs bins in which the new \gones and \afones have been extracted. The new low \qsqs measurements of the moments evaluated both with and without model inputs for the unmeasured kinematic regions were then compared with various \chipts calculations, phenomenological predictions and past measurements, particularly the EG1b or DIS data whenever applicable.


%======================= Integrals observation, conclusions
%It must be noted that the 
The EG4 results provide the only data points in the very low \qsqs region ($Q^2<0.05$ GeV$^2$) where \chipts is thought to be able to make rigorous calculations. The high precision data will provide important benchmarks for the future calculations in this kinematics. In the case of the first moment $\Gamma^d_1$, the EG4 results show remarkable agreement with the latest \chipts prediction by Bernard \etal \cite{BEKM13} in the very low \qsqs region. The phenomenological predictions which have much larger \qsqs coverage also seem to agree within the uncertainties of our measurements, with the predictions of Soffer \etal showing slightly better comparison than others. 
Likewise, the very low \qsqs results of the generalized GDH integral $\bar{I}_{TT}$ are indeed observed to converge towards the GDH sum rule and thus getting very close to the \chipts predictions by Bernard \etal \cite{BEKM13}. 
%However, only one or two higher order terms can be calculated confidently, since higher orders require additional (unknown) constants. Therefore, \chipts predictions do reasonably well at ultra-low \qsqs but cannot be expected to work at the higher \qsq, where the data show a turn-around and a transition towards positive values.
Finally, the generalized forward polarizability ($\gamma^d_0$) for the deuteron calculated from the EG4 data and the \chipts calculations by Bernard \etal and Kao \etal seem to converge at the lowest \qsqs bins.  However, the \chipts based predictions from Pascalutsa \etal and the MAID prediction seems to be well off the current results for all three moments.
%=======================


% Outlook
The deuteron data in combination with the EG4 proton data taken under similar conditions %by the same experiment 
(currently being analyzed by another collaborator) % and results expected to come very soon)
will be useful in extracting neutron quantities in the near future, which is valuable because of the unavailability of free neutron targets. Moreover, due to the complexities of the nuclear medium effects, neutron data from deuteron will be very important to enhance confidence in similar neutron results extracted from other nuclear targets - particularly $^3$He. 

The work presented in this analysis has improved our understanding of the field of the nucleon spin structure and contributed to more solid foundation for future advancements. 
The new data on spin structure functions and their moments will help various \chipts calculations and phenomenological models such as MAID to better constrain their parameters enabling them to make better predictions in the future.  %Order these two sentences changed from the following ones w.r.t 5th chapter
With the availability of the high precision data in the previously (largely) unmeasured region that has the potential to help constrain the theories and models, it is hoped that a unified description of spin structure functions over all kinematic regions will be possible in the future.

