\linenumbers

\chapter{Introduction}
\label{cha:Introduction}

%This chapter/section is about Introduction and Theory
%Created by kp on March 29, 2011


% 10/19/13: pg 4 of the following paper has a very nice quick upshot/overview of nucleon spin structure:  Dropbox/LatestReads/PapersRd_GDHetc/SomeConcepts/axialCurrentsgA_gA8_gA0_spinContent_BarquillaEtal0303020.ps.gz.pdf
% 10/20/13: pg 45 of V. Sulkosky thesis: Spin crisis resolved by taking into account the gluon spin, the sea quarks (q-q_bar pairs) and the angular momentum of partons
% Study the nucleon structure to shed new light on non-perturbative QCD. Slide2: http://ipnweb.in2p3.fr/~semin/semin_ipn/talks/moutarde_ipn_2013_03_25.pdf

% Also use introductory paragraph from my Oral-Qual writeup (including the figure as well may be)
% How about a picture of the family of quarks and leptons
% How about a picture of the running of coupling constant?

\hspace{0.5cm}
% \newline \newline

%\section{Introduction}
%\textcolor{red}{Colored texts are either temporary}.

A truly vast amount of data on the inelastic structure of the nucleon%s 
has been accumulated since the late 1960s %over the past 40 years 
from both fixed target and colliding beam experiments with polarized as well as un-polarized incident photons, (anti)electrons, muons and (anti)neutrinos as well as (anti)protons on a variety of targets (both polarized and unpolarized) from hydrogen through iron \cite{KuhnHUGS}. %\footnote{Photon and lepton scattering has been the predominant and very powerful methods to probe the composite systems like nucleons, nuclei or even atoms for three primary reasons: (a) photons and leptons are point-like particles with no known substructure or excited states; (b) the underlying electro-weak interaction is well understood; (c) the interaction is sufficiently weak (so they can penetrate deeply into the target without disturbing its substructure, thus enabling the extraction of the internal structure of the target with a relatively easy interpretation following a perturbative treatment.)} % SEK comment: Too long for a footnote: Put into text or refer to later
%\footnote{Lepton scattering has been a powerful and predominant method to probe composite objects for three primary reasons: (a) leptons are structureless point-like particles; (b) the underlying electro-weak interaction is well understood; (c) the interaction is sufficiently weak (to probe the target deeply without too much disturbance), allowing relatively easy perturbative treatments to extract and interpret the internal structure.} % SEK comment: Too long for a footnote: Put into text or refer to later
 The initial measurements at SLAC confirmed the picture of the nucleon as made up of partons (now identified with quarks and gluons). Since then more precise measurements have been conducted at several accelerators, improving our knowledge and understanding about the nucleon structure (both spin-dependent and spin-averaged), and, at the same time, continuing to give us new and sometimes very surprising results such as the original ``European Muon Collaboration (EMC)-Effect'' \cite{Aubert1983275}, % \footnote{EMC=European Muon Collaboration}, %http://www.sciencedirect.com/science/article/pii/0370269383904379
 the violation of the Gottfried sum rule \cite{PhysRevLett.18.1174, PhysRevD.50.R1}, and the so-called "Spin-Crisis"  \cite{Ashman1988364, Piguet:1989va} (see below). %  even giving some hints that quarks might have substructure \cite{KuhnHUGS}. (SEK: 12/2/13: No longer up-to-date)
\textcolor{red}{Will soon work on the order of references/citations.}

%\section{Spin Structure of Nucleons}
With such a vast amount of experimental data available%from DIS experiments
, a lot is now known about the spin-averaged quark structure of the nucleon, but a lot less is known about the spin-structure of the nucleon in terms of its constituents – quarks and gluons \cite{KuhnHUGS}. In a simple non-relativistic model one would expect the quarks to carry the entire spin of the nucleon, but one of the early more %rather SEK
realistic theories that explained the partonic substructure of the nucleon, the Naive %Naïve 
Parton Model (NPM), predicted that 60\% of the nucleon spin is carried by the quarks \cite{vipuliTh}. %http://www.sciencedirect.com/science/article/pii/0370269388915237 Vipuli used this reference, so perhaps better for me too

%After the %GED
The polarized beam and target technologies have greatly advanced during the last three decades, and %and, GED
many subsequent experiments on nucleons and some nuclei have contributed to the extraction of their spin structure functions \gone and \gtwo, which carry information on how the spin is distributed inside the target. %are related to the spin carried by the quarks in the nucleon. 
One of the first experiments carried out at SLAC, in a limited kinematic region, seemed to confirm the predictions of the NPM. However, a subsequent, more precise measurement over a % at a %GED
larger kinematic region performed by the EMC experiment at CERN reported that, contrary to the NPM predictions, only $12\pm 17\%$ %(i.e., practically none) 
of the spin is carried by the quarks \cite{Ashman1988364, Piguet:1989va}. This discovery of the so-called ``spin crisis'' %has
 sparked a large interest in measuring the spin content of the nucleon, giving birth to several experiments (completed, underway and proposed) around the globe. The %subsequent SEK cor.
 theoretical developments of Quantum Chromodynamics (QCD) - the quantum field theory that describes the nuclear interaction between the quarks and gluons - have clarified our picture of the nucleon spin structure in great details. With the discovery of a unique QCD property known as ``asymptotic freedom'', quarks are %now SEK cor.
 known to be essentially free %and interact very weakly at higher energies (or shorter distances)  %ADeur
at high energies (typically several GeV)
allowing perturbative QCD (PQCD) calculations of testable predictions for processes involving high energy or high momentum transfers \cite{wikiCoupling}.  
Verifying Bjorken sum rule, %The so-called Bjorken sum rule,
\textcolor{red}{Reference to be added soon}
which relates results from %of the %SEK cor
 inclusive, polarized deep inelastic scattering (DIS) % lepton-nucleon scattering (a high energy process) 
to the %fundamental 
axial coupling constant $g_A$ of neutron beta decay, is a precise %such SEK cor
 test of QCD in its spin sector. %The interpretation of existing DIS results has %deep inelastic %GED
The DIS results have verified the Bjorken sum rule at the level of 10\% accuracy and has shown that only about $30 \pm 10\%$ of the nucleon spin is carried by the quarks; the rest of the spin must reside either in gluons or orbital angular momentum of its constituents. Experiments to measure the gluon contribution are underway at %DESY, SEK cor
Brookhaven National Laboratory (BNL) and CERN. % \cite{vipuliTh}. GED

%To make the statements more sensible (use ideas from K Kramer thesis pg 45) and also get some more perspectives on g2 and higher twist terms
Probing nucleon structure %at %SEK cor
on the other end %the other side %GED
of the energy scale (i.e. probing with low momentum transfers) %(\qsq)) %SEK cor
 provides information about long distance structure, % of the target, 
which is also associated with static properties of the nucleon. In this low energy regime, however, QCD calculations with the established perturbative methods become difficult or even impossible because %of
 the strong coupling %constant 
($\alpha_s$)  becomes %very 
large.% \footnote{The dependance of the QCD coupling constant ($\alpha_s$) on the energy scale is known as "running of the coupling". The earlier mentioned high energy property of QCD - asymptotic freedom - is a consequence of this running phenomenon, just as another famous property called "confinement" at low energies or large distances \cite{wikiCoupling}. }
% , and so
Thus perturbative expansions %(in powers of $\alpha_s$) 
do not converge. %In other words, in this 
In this energy regime, the partons become very strongly coupled to the point of being confined into hadrons which now emerge as % the new (effective) 
the effective degrees of freedom for the interaction. Therefore, other methods must be relied on to make predictions in these non-perturbative energy scales. 
For example, %at very low energies, 
effective theories such as chiral perturbation theory (\chipt) are used.
 There is also an %a small %SEK cor
  intermediate region where neither of these approaches (PQCD or \chipt) %(i.e. of PQCD and \chipt) %SEK cor
  is expected to work. In this region, it is %has been %SEK cor
  expected that lattice QCD methods will provide testable predictions in the near future. There are also some phenomenological models aimed at describing the entire kinematic range. The description of the low energy regime in terms of these theories and models is %still a challenge 
challenging
and theories used here %are still fraught with several issues (see Chap. \ref{Chap2}). %Despite the existing issues and challenges, there %SEK cor
still face difficulties (see below).%Chap. \ref{Chap2}).
  There are several %calculations and 
  predictions (for both nucleons as well as some light nuclei %targets %GED
  such as the deuteron and Helium-3) from these low energy theories and models on various   observables %  observable/measurable quantities, 
  which can be %put to test %GED
  tested %using %the %SEK cor
  experimentally. Therefore, having high precision data at the lowest possible momentum transfer %transfers %\qsq %SEK cor: Q2 not defined yet
  is %very important 
  necessary to test these %already available 
  predictions. In addition, new results will also help constrain %the %SEK cor
  future calculations and %address the various issues plaguing them.  %SEK cor %kp: thus contributing to the further understanding of the trasition from partonic to hadronic descriptions of strong interaction. 
  provide input for detailed corrections to higher energy data.

%The order of the following three topics may change or mix into one or two
%Brief discussion on what will be measured and tested with theory/models. 
%\textcolor{red}{Should I also have to introduce the integrals that we're going to measure and compare with the predictions? %SEK cor: No
%Or, is it enough to have them discussed in the next bigger "Theory" chapter?}. %SEK com: Yes

%EG4 %Significance of deuteron data.
With that perspective and motivation, the ``EG4'' experiment (E06-017) % Proton proposal is (E03-006)
for a precision double polarization measurement at low momentum transfer %s %(down to about $Q^2=0.2 ~ GeV^2$)%SEK cor
  using both proton and deuteron targets and the Hall B CLAS detector was performed at Jefferson Lab. 
In addition to the usefulness of the measured %low \qsq %SEK cor
deuteron data for testing %the 
theoretical predictions calculated for the deuteron itself, the data %will also be useful to extract %GED
are also necessary for extracting neutron data%\footnote{Due to the relatively very short lifetime and various other complexities, no free neutron target has been %designed and %SEK
%devised yet. All the relevant neutron information so far has come from measurements on nuclear targets (mostly very light nuclei such as $^2$H and $^3$He).} 
 in combination with %the  %SEK cor
similar data from the proton target. %experiments. %SEK cor
An experiment with the similar %goals of getting neutron data %GED
goal of probing the neutron at low momentum transfers but using $^3$He was performed in Hall A \cite{propE97_110}. %However, due to the not-fully-understood complexities of nuclear medium effects, neutron information extracted from only one type of nuclear target cannot give us enough confidence in our measurements. So, having results from different types of targets is very important for better confidence in the extracted neutron results, which will enable us %provide the benchmarks to test the similar %GED
%to test the theoretical and model predictions as is done %GED added "as is done" 
%for other targets (deuteron, proton). 
However, to be able to control nuclear medium effects, neutron information must be extracted from both $^3He$ and $^2D$.
The data on the deuteron (and eventually on the neutron) will not only be useful to test the theoretical predictions at low but non-zero %finite 
momentum transfers but they can also be %used to extrapolate 
extrapolated to the % the 
real photon %absorption %scattering ($Q^2=0$) 
limit, thus %providing tests of 
testing some long standing predictions such as the Gerasimov-Drell-Hearn (GDH) sum rule \textcolor{red}{Reference to be added soon} %(derived not from the aforementioned low energy effective theories but independently from general principles). %The EG4 experiment has also collected proton target data under similar experimental conditions and the 
derived from general principles.
The analysis of the deuteron data is the subject of this note. %and the proton target data collected by EG4 are being analyzed by another member of the collaboration. 

%GED suggested a new paragraph
In the future, we will extract information %about the neutron 
from the % The proton target data are also being analyzed by the other members of the collaboration. Once the neutron data is extracted from the deuteron data and proton data from the same experiment is also ready for comparison, the combined data from the two different targets will 
deuteron and proton data from the EG4 experiment to %GED
provide a self-consistent determination %test 
of the Bjorken sum, % rule, % which is thought to be the best rule to be tested experimentally in an effort to understand the trasition from the partonic to hadronic descriptions of the strong interaction. %SEK: not true.
helping us to understand the transition from the partonic to hadronic descriptions of the strong interaction. The data will also be useful in studying the validity of %the %GED
quark-hadron duality in the spin sector, thus helping further to understand the transition from the partonic to hadronic pictures. %http://arxiv.org/abs/hep-ph/0501217

%So four main motivations
% 1) Testing low q2 models (Chipt,phenom, LQCD, higher twist) (see Nevzat pg 27)
% 2) Testing Real photon predictions such as GDH
% 3) Testing Duality
% 4) Testing Bjorken rule



\section{Inclusive Electron Scattering}
\label{inclScatKineVars}%Sulkosky points: (Walecka book)
High energy particle scattering processes %, in a way, 
provide very powerful microscopes to examine %the tiny 
objects such as nuclei and nucleons. % which are so tiny that they are absolutely inaccessible from the reach of traditional microscopes. %And, the scattering 
Scattering of leptons (most commonly electrons) is one of the most extensively used processes. For example, the scattering of high energy leptons off nucleons has played a key role in determining the partonic structure of the nucleons. Following are some of the advantages of lepton (and in particular electron) scattering: % Even though each of the lepton types have its own advantages over others, electron scattering has been the one that has been used the most world over because of the reasons as listed below. 

%Also remember the following from the introductory Chapter 1
%\footnote{Photon and lepton scattering has been the predominant and very powerful methods to probe the composite systems like nucleons, nuclei or even atoms for three primary reasons: (a) photons and leptons are point-like particles with no known substructure or excited states; (b) the underlying electro-weak interaction is well understood; (c) the interaction is sufficiently weak (so they can penetrate deeply into the target without disturbing its substructure, thus enabling the extraction of the internal structure of the target with a relatively easy interpretation following a perturbative treatment.)} % SEK comment: Too long for a footnote: Put into text or refer to later

%Why electron scattering? \textcolor{red}{pgW, may be in footnote} \\
%\textcolor{red}{(Should I put the list on footnote, just not have it at all?)} 
\begin{itemize} 
\item Leptons %Electrons 
interact through the electroweak interaction which is very well understood. %(via Quantum Electrodynamics (QED) - (one of) the best tested theory in Physics) (This is also true for any other charged lepton). \textcolor{red}{(citations)} %\\% pg 30
\item The interaction is relatively weak, thus %enabling measurements with only %very 
%small disturbances to the target structure. %(true for any lepton). %\\% pg 30
allowing the use of perturbative QED.
\item In electron scattering, one can, moreover, control and vary the polarization of the virtual photon (exchanged during the interaction) by changing the electron kinematics. This allows the separation of the charge and current interaction. %In sum, electron scattering gives rise to a precisely defined virtual quantum of electromagnetic radiation, and hence electrons provide a precision tool for examining the structure of nuclei and nucleons. %\\%pg 32
Data from the scattering of polarized electrons by polarized targets allows one to examine the target's strong-interaction spin structure.
\item A %An additional 
great advantage of electrons is that they can be copiously produced in the laboratory relatively easily and at low costs, and since they are charged, they can readily be accelerated and detected. (It is not as easy and cheap to produce and handle 
%the other particles such as other lepton types or hadrons). %\\%pg 32
the other lepton types.
%\item In addition, electron scattering is a rather versatile tool. We know from the theory of electromagnetism that, just as two charges, two currents also interact with each other (as magnetic interaction). Since, a moving charge is equivalent to a current, the electron beam has both charge and current and hence, it interacts with the target both through the Coulomb interaction  and the magnetic force, thus giving us an opportunity to get more information than one would get from a single interaction. %\\%pg 32/33
%Corrolary???:    * electron scattering measures the full transition matrix element of the target current. In addition, with electron scattering one has the possibility of bringing out high multipoles of the current at large values of κR.
%\item Furthermore, the interference between $\gamma$ and $Z_0$ exchange, where $Z_0$ is the heavy boson mediating the weak neutral current interaction, gives rise to mostly small but sometimes very significant parity violation effects in certain measurements. For example, one measure of parity violation is the asymmetry arising from the difference in cross section of right- and left-handed electrons in inclusive electron scattering. Parity violation in (e, e') doubles the information content in electron scattering as it provides a means of measuring the spatial distribution of weak neutral current in nuclei and nucleons. %\\%pg 33/34
%\item  Finally, data from the scattering of polarized electrons by polarized targets allows one to examine the target's strong-interaction spin structure. %\\  [Hu83]
\end{itemize}

In this section, we discuss the process of inclusive electron scattering (in which only the scattered electron is detected ignoring the rest of the components of the final state after the interaction). % is discussed. 
In doing so, the relevant kinematic variables and related physical quantities to be measured or calculated from the process will be introduced and %, if need be, %if pertinent, 
some of their relations with one another will be deduced and discussed. 




\subsection{Kinematic Variables}
A lepton scattering process, in which an incoming lepton represented by $l(p)$ of four momentum $p=p^{\mu}=(E,\vec{k})$ scatters off a target $N(P)$ which is usually a nucleon or a nucleus at rest and with four momentum $P=P^{\mu}=(M,\vec{0})$, can simply be represented by
\begin{eqnarray}
\label{lepScat}
l(p) + N(P) \rightarrow l(p') + X(P')
\end{eqnarray}
where $l(p')$ and $X(P')$ represent the scattered lepton and the rest of the final state (which can have any number of particles) with four momenta $p'^{\mu}=(E',\vec{k'})$ and $P'^{\mu}=(E_X,\vec{k_X})$ respectively. The scattering angle which is the angle between the incident and outgoing path/direction of the electron is denoted by $\theta$. The final (hadronic) state denoted by $x$ is %ignored
not measured, with only the scattered electron detected and measured by the detector(s). In the first order (Born) approximation of the process, a virtual photon is exchanged (as depicted in Fig (\ref{figInclSc})) whose four momentum is equal to the difference between that of the incident and the scattered electron and is given by $(p-p')^{\mu}=(\nu,\vec{q})$, where $\nu = (P.q)/M $ and $\vec{q}$ represent the energy and 3-momentum transferred by the incident electron to the target $N(P)$.

\begin{figure}[h]%[hp]%[htpb]
\centering
	%\leavevmode \includegraphics[scale=0.65]{Chap2Theory/Figures/inclusiveScat.eps}
	\leavevmode \includegraphics[scale=0.65]{TexmakerMyFinTh/Chap2Theory/Figures/inclusiveScat.png}
	\caption[Inclusive scattering (Born Approximation)]{Lowest order (Born approximation) Feynmann diagram representing the process of inclusive lepton scattering}
	\label{figInclSc}
\end{figure}

The kinematics of the scattering process, for a given beam energy E, can be completely described in terms of %any 
two of the following Lorentz invariant variables. %and the scattering angle. 
\begin{eqnarray}
\label{nuQ2W}
\nu &=& E - E'    \\
Q^2 &=& -q^2  \simeq  4 E E' sin^2 \frac{\theta}{2}   \\
W   &=& \sqrt{(P+q)^2} = \sqrt{M^2 + 2M \nu  - Q^2}          \\
x   &=& \frac{Q^2}{2P\cdot q} = \frac{Q^2}{2M\nu}     \\
y   &=& \frac{q \cdot P}{p \cdot P} = \frac{\nu}{E}
\end{eqnarray}
where $Q^2 = - q^2$ is the negative of the squared four-momentum transferred (with electron mass neglected %(compared to the target mass) 
in the expression for \qsq), which defines the resolution %strength 
of the electron probe;  $W$ is the invariant mass of the unmeasured final state ($x$);  $x$ is known as the Bjorken scaling variable, which is also interpreted as the momentum fraction carried by the struck quark (parton) in the infinite momentum frame;  $M$ is the nucleon mass $\approx 0.939$ GeV, and lastly, $y$ is the fraction of the energy that is lost by the lepton during the process. 
 



\subsection{Differential Cross Section and Structure Functions}

The differential cross section for the process of inclusive (polarized) electron %lepton 
scattering on (polarized) targets can be expressed, in the Born approximation, in terms of four dimensionless structure functions $F_1(x,Q^2)$, $F_2(x,Q^2)$, $g_1(x,Q^2)$, and $g_2(x,Q^2)$, effectively parameterizing the internal hadronic structure information into four response functions. % - two spin independent ($W_{1,2}$) and two spin dependent ($G_{1,2}$) function.
For example, in the case of the anti-parallel or parallel beam and target polarizations, the spin-dependent (polarized) inclusive cross sections can be expressed as follows:

\begin{comment}
\begin{eqnarray}
\label{diffXS2}
%\frac{d^2\sigma}{d\Omega dE'} = \left(\frac{d\sigma}{d\Omega}\right)_{Mott}
\frac{d^2\sigma}{d\Omega dE'} = \left(\frac{d\sigma}{d\Omega}\right)_{Point}
 \left( \frac{2}{M}F_1(x,Q^2) tan^2 \frac{\theta}{2} + \frac{1}{\nu} F_2(x,Q^2) \right)
\end{eqnarray}

%Followign also worked but I chose to use the eqnarray abelow
\begin{equation*}
  \begin{aligned}
    \frac{d^2\sigma}{d\Omega dE'}  
    = \left(\frac{d\sigma}{d\Omega}\right)_{Point}  
    \Bigg[ &  %\left[ and \right] couldn't be put on two sides of the line breaker '\\'
       \frac{2}{M} F_1(x,Q^2) tan^2 \frac{\theta}{2} + \frac{1}{\nu} F_2(x,Q^2) \\
     & \pm  2\, tan^2 \frac{\theta}{2} 
     \left[   (E + E' cos\theta) \frac{M^2}{\nu} g_1(Q^2,\nu) - \gamma^2 M^2 g_2(Q^2,\nu)    \right]
     \Bigg]
  \end{aligned}
\end{equation*}

\end{comment}



%\begin{eqnarray*} % The start '*' turns off the equation numbering
\begin{eqnarray}
  \begin{aligned}
    \frac{d^2\sigma}{d\Omega dE'}  
    = \left(\frac{d\sigma}{d\Omega}\right)_{Point}  
    \Bigg[ &  %\left[ and \right] couldn't be put on two sides of the line breaker '\\'
       \frac{2}{M} F_1(x,Q^2) tan^2 \frac{\theta}{2} + \frac{1}{\nu} F_2(x,Q^2) \\
     & \pm  2\, tan^2 \frac{\theta}{2} 
     \left[   (E + E' cos\theta) \frac{M^2}{\nu} g_1(Q^2,\nu) - \gamma^2 M^2 g_2(Q^2,\nu)    \right]
     \Bigg]
  \end{aligned}
\end{eqnarray}
%\begin{eqnarray*} % The start '*' turns off the equation numbering

where ``+'' refers to anti-parallel beam helicity and target polarization, while ``-'' refers to
the parallel case.  
And the Point %Mott 
cross section (for the lepton scattering from a Dirac particle - a spin-1/2 point particle of charge +e) given by
\begin{eqnarray}
\label{MottXS}
%\left(\frac{d\sigma}{d\Omega}\right)_{Mott} 
\left(\frac{d\sigma}{d\Omega}\right)_{Point} = \frac{\alpha^2 cos^2 \frac{\theta}{2}}{4E^2 sin^4\frac{\theta}{2}}
\frac{E'}{E}
\end{eqnarray}
with $\frac{E'}{E}$ being the recoil factor.



These kind of relationships allow the measurement of structure functions by measuring cross-sections corresponding to different combinations of %the 
beam and target polarizations. For example, one can extract the first two structure functions $F_1$ and $F_2$ from the unpolarized scattering experiments, %because the total spin averaged differential cross section in the lab frame is related to the these unpolarized structure functions as follows:
whereas, the spin structure functions $g_1$ and $g_2$ can be measured in experiments with polarized elctron beam and polarized targets
and by measuring the cross section difference between the anti-parallel and parallel beam-target polarizations.



%\section{Some Integrals/Moments and Sum Rules of \gone}
\section{Moments  of \gones and Sum Rules}
\label{integrals}
%\section{Moments of Structure functions and Sum Rules}
Moments of structure functions are their integrals (over the complete x range) weighted by various powers of the variable x. The $n^{th}$ moment of \gones, for example, is given by 

\begin{eqnarray}
\label{GammaN}
\Gamma_n(Q^2)  = \int^{1}_0  g_1(x,Q^2) x^{(n-1)}  dx
\end{eqnarray}

The moments allow the studies of the (\qsqs dependence of) fundamental properties of nucleon %target 
structure. For example, the first moment of $x F_1$ %\textcolor{red}{It seems you corrected $F_1$ to $F_2$. Should it really be $F_2$ instead?}
of a nucleon gives the total momentum or mass fraction carried by quarks and the first moment of \gones gives the fraction of the nucleon spin contributed by the quark helicities. %Moments and their related integrals provide powerful tools to study the fundamental properties of nucleon structure. %SEK redundant
These integrals get their particular significance from the fact that they can be predicted from rigorous theoretical methods, such as in the sum rules derived from general assumptions or from the method of Operator Product Expansion, lattice QCD calculations and \chipts calculations \footnote{In contrast, the same is not true about the structure functions because presently their complete description %and derivation or prediction 
based on QCD first principles %is very difficult and is not available as yet. %till date.
has not been possible yet (especially in the low to intermediate momentum transfer regions due to the strong coupling property of QCD).} (see Sec. \ref{theoryTools}). %Among the infinite set of moments and integrals, three
Their importance can be highlighted from the fact that it was the experimental tests of the sum rules involving the first moments of nucleon that led to the discovery of the original ``spin crisis'' and provided a significant test of QCD in the spin sector  \cite{pLeaderKuhnChen}. %Through the detailed measurement of the \qsqs evolution of the Bjorken sum rule (difference of the first moments of proton and neutron). KCL (pg 43): The Bjorken sum rule is quite rigorous, involving only the asumption of isospin invariance, and insspired Feynman to say that its failure would signal the demise of QCD.
%KCL pg 46: As already mentioned in Section 2.1 relations between moments of structure functions and matrix elements of operators (kp: as derived from OPE) are only valid if the moments include the elastic contributions located at x = 1. In the deep inelastic region the elastic contributions are negligible and are not included in experimental estimates of the moments, but at low Q 2 , in the resonance region, the distinction is important. Thus moments in the latter region, without an elastic contribution will be labeled \bar{Gamma}.

In this section, three integrals are considered which have been calculated from the EG4 data on the deuteron - the first moment of \gones ($\Gamma_1$), the generalized GDH integral ($\bar{I}_{TT}$), and the generalized forward spin polarizability ($\gamma_0$). 
%Study of various structure functions provide some deep insights into the structure of the target particles.

%The product \afones is proportional to the virtual photoabsorption cross section $\sigma_{TT}$ and, therefore, directly enters sum rules for the photon point.


%\label{goneMoments}
\subsection{First moment $\Gamma_1$ of \gone }
The first moment of \gones is the integral of \gones over the complete range of the Bjorken scaling variable x.
\begin{eqnarray}
\label{eqGamma1}
\Gamma_1(Q^2)  = \int^{1}_0  g_1(x,Q^2) dx
\end{eqnarray}

This moment gives, in the quark-parton model, the fraction of the nucleon spin contributed by the quark helicities and enters directly into two historically important sum rules - Ellis-Jaffe sum rule and Bjorken sum rule. Measurements of the moment on the proton by the European Muon Collaboration (EMC) in 1988 showed that the Ellis-Jaffe sum rule is violated, which meant that the long held belief that all the proton spin is carried by quarks is not %no longer %SED
true, thus, sparking the well known ``spin crisis''. On the other hand, measurements from SLAC, CERN, Fermilab, DESY, and more recently, from JLab, have confirmed the Bjorken sum rule (which relates the difference of the first moments of the proton and the neutron to the fundamental axial coupling constant ($g_A$) of neutron beta decay) at the level of 10\% accuracy, thus helping establish the QCD as the correct theory of the strong interactions. The moment also enters into the virtual photon extension of another famous sum rule -  the GDH sum rule (see below).

In addition, the moment is studied on its own right because it provides a powerful tool to test the validity of various theories and models in which it is calculable. In the past, it has been measured on proton, deuteron and neutron ($^3$He) at SLAC, CERN and DESY in the DIS region in order to understand the quark spin contribution as well as to test the validity of the Bjorken sum rule and hence %the %SED
QCD as a result \cite{pLeaderKuhnChen}. Recently, it has also been measured at JLab from DIS down to a fairly low \qsqs region. In the intermediate and low momentum transfers, some phenomenological model predictions are available, whereas in the very low \qsqs region, several % many %different %SED
chiral perturbation theory (\chipt) calculations are available. %Fig. \ref{modelGm1} shows some of these calculations along with the past measurements from SLAC and from the EG1b experiment at JLab.

\begin{comment}
\begin{figure}[h]%[htpb] %ht, htpb (p - float, b = bottom, h=? t = top)
  %\leavevmode \includegraphics[width=1.0\textwidth]{Chap2Theory/Figures/integralsModelsDataNoEG4Gm1LowQ2nwRange.eps} 
  \leavevmode \includegraphics[width=1.0\textwidth]{TexmakerMyFinTh/Chap2Theory/Figures/integralsModelsDataNoEG4Gm1LowQ2nwRange.png} 
  \caption[Predictions for $\Gamma^d_1$ and some data]{Some theoretical predictions for $\Gamma^d_1$ and some data from past measurements. The theories and models which make these predictions are described in Sec. \ref{theoryTools}.}
  \label{modelGm1}  
\end{figure}
\end{comment}	
	
	

%\input{Chap2Theory/theoryGDH1.tex}
\pagebreak   %12/6/13 (because nothing else that I tried worked)
\subsection{Generalized GDH Integral}
\subsubsection{GDH Sum Rule}
\label{gdhSmRl}
% 10/13/13: (Downloaded from secure page) mash it up with what's written already
%Bring in the picture of a magnet (as used in one of my posters or S6 of ODUsumrTalk10_lptp.ppt   (Sukosky thesis, pg 40)
The Gerasimov-Drell-Hearn (GDH) sum rule \cite{GDHsumRule0,GDHsumRule} relates the energy weighted sum of a particle's photo-absorption cross sections to its anomalous magnetic moment $\kappa$. For a target of arbitrary spin S, the sum rule is:
\begin{eqnarray}
\label{eqGDHsmRl}
\int^{\infty}_{\nu_{th}}  \frac{\sigma_P(\nu) - \sigma_A(\nu)}{\nu} = -4 \pi^2 \alpha S(\frac{\kappa}{M})^2
\end{eqnarray}
where $\sigma_P$ and $\sigma_A$ are the photoabsorption cross sections with photon helicity parallel and anti-parallel to the target spin respectively. M and $\kappa$ represent the target mass and anomalous magnetic moment respectively and S represents the target spin. %This sum rule is for real photon scattering (i.e., \qsq=0)%, which relates the static property (here "anomalous magnetic moment") of the ground state of a given particle to the sum of the dynamic properties (cross-sections for all the excitations (i.e., all the scattering states except for the elastic one, in other words, for which the 
The integration extends from the onset $\nu_{th}$ of the inelastic region \footnote{The pion photo-production threshold given by $\nu_{th} = m_{\pi}(1+ m_{\pi}/2M) \approx 150 MeV$ marks the onset of the inelastic region for the nucleons, but for nuclei, the summation starts from the first nuclear excitation level} through the entire kinematic range and is weighted by the inverse of the photon energy $\nu$. 

\begin{comment}
The sum rule was derived in the late 1960s based on some very general assumptions as follows:
\begin{enumerate}
\item \textbf{Lorentz and gauge invarianc}e in the form of the \textbf{low energy theorem of Low, Goldman and Goldberger} %citation from K. Slifer (pg 47)
\item \textbf{Unitarity} in the form of the \textbf{optical theorem}
\item \textbf{Causality} in the form of an \textbf{unsubtracted dispersion relation for forward Compton scattering}. %citation
\end{enumerate}
\end{comment}

%Following numbers from my first ever poster (on EG4 topic)
%Proton = -205 mb (к_p =µ_p /µ_N -1=1.79)
%Neutron = -233 mb (к_n =µ_n /µ_N = - 1.91)
%Deuteron = -0.65 mb (к_d =µ_d /µ_N -1=- 0.143)
The sum rule for the proton has been measured (at various places such as Mainz, Bonn, BNL %SLAC %SEK
and others) and verified to within 10\% \cite{PhysRevLett.91.192001, PhysRevLett.94.162001, Drechsel:2004ki, PhysRevLett.102.172002} and some deuteron results exist from Mainz and Bonn, but there is very little or no data available on neutron and other targets;%, more so in the very low \qsq region.

\paragraph{Implications of the sum rule} 
The sum rule relates the static property $\kappa$ of a particle's ground state with the sum of the dynamic properties of all the excited states. One deeper significance of this sum rule is that if a particle has a non-zero anomalous magnetic moment, then it must have some internal structure, and, therefore, a finite size, in order to have the excited states (a point-like particle cannot have excited states).
% 10/15/13
%The GDH sum rule relates a particles's anomalous magnetic moment to an energy weighted sum over its photo absorption cross sections. The rule is valuable because it connects the static ground state properties of a particle with the dynamic properties of its excited states. One important implication of the validity of the rule is that if a non-zero anomalous magnetic moment exists for a given particle, then the particle must have internal structure and a finite size because a structureless point particle cannot have excited states, thuse violating the sum rule. 
Because of the same reason, the discovery of nucleon anomalous magnetic moments provided one of the first strong indications that the nucleons had some intrinsic internal structure.

%The sum rule implies that if a particle has a non-zero anomalous magnetic moment, then the particle has some structure and is not a point-like particle. This is because, to have non-zero right hand side, one must have non-zero value of the left hand side of the sum rule, which means the particle has some excited states. And, for a particle to have excited states, it must have some internal structure.

In addition to the benefit of that implication, the sum rule and its extension to $Q^2>0$ provides an important testing ground for various theoretical predictions based on QCD and its effective theories/models.








%\subsubsection{Sum Rule for Spin-1/2 Particle and for an Arbitrary Target}

\subsubsection{Generalization of the GDH Sum (Rule)}

In order to investigate the ``spin crisis'' of the 1980's, Anselmino \etal \cite{AnselIoffeLeader} proposed that the real photon (\qsq=0) GDH integral could be extended to electroproduction cross sections (finite \qsq) and that the experimental 
determination %test %SEK
of the extended integral would shed light on the transition from the perturbative to non-perturbative QCD. The idea was to use the virtual photoabsorption cross sections in place of the real photoabsorption cross sections and proceed in exactly the same way as when deriving the real photon GDH sum rule. This extension depends somewhat % Depending %SEK
on the choice of the virtual photon flux %(see Sec. \ref{vpcs}), 
and on how the spin structure function \gtwos is considered \cite{propE06_017}. In one extension the virtual photon flux given by $K=\nu$ %(see Eq. \ref{vPhFluxA}) 
is chosen and the real photoabsorption cross section difference in Eq. \ref{eqGDHsmRl} are 
is replaced by the corresponding virtual photoabsorption cross section difference $2 \sigma_{TT}$% as given by Eq. \ref{sigTnTT}
. %With the use of Eq. \ref{sigTT}, and some algebraic manipulation, 
As a result, we get the following extended GDH integral (considering only the inelastic contribution starting from the pion production threshold) \cite{pLeaderKuhnChen}

\begin{eqnarray}
\label{eqGenGDH}
\bar{I}_{TT} = \frac{2M^2}{Q^2} \int_{0}^{x_0(Q^2)} dx [ g_1(x,Q^2) - \frac{4M^2x^2}{Q^2} g_2(x,Q^2) ]
\end{eqnarray}

where $x_0(Q^2) = Q^2/(Q^2 + m_{\pi}(2M+m_{\pi}))$ is the pion production threshold that defines the onset of the inelastic region.

%Using Eq. \ref{vPhAsym}, the 
The integral can also be expressed in terms of the first moment of the product $A_1F1$ as follows:
\begin{eqnarray}
\label{eqGenGDHaf}
\bar{I}_{TT} = \frac{2M^2}{Q^2} \int_{0}^{x_0(Q^2)} dx A_1(x,Q^2)F_1(x,Q^2)
\end{eqnarray}

where $A_1$ is the virtual photon asymmetry as given by:

\begin{eqnarray}
\label{vPhAsym}
A_1 (x,Q^2) &=&  \frac{\sigma^T_{ \frac{1}{2} } - \sigma^T_{ \frac{3}{2} } }{ \sigma^T_{ \frac{1}{2} } + \sigma^T_{ \frac{3}{2} }  }  
=  \frac{g_1(x,Q^2) - \gamma^2 g_2(x,Q^2)}{F_1(x,Q^2)}         \\
%\label{vPhAsymA2}
%A_2 (x,Q^2) &=& \frac{2 \sigma^{TL}_{ \frac{1}{2} } }{ \sigma^T_{ \frac{1}{2} } + \sigma^T_{ \frac{3}{2} }  }  
%= \frac{\gamma [g_1(x,Q^2) + g_2(x,Q^2)] }{F_1(x,Q^2)}  
\end{eqnarray}


\begin{comment}
Fig. \ref{modelItt} shows a \chipts prediction along with the integral calculated from the model used in the EG4 data analysis covered by this note (see below). As is evident from the figure, the limiting value of the integral as \qsqs goes to zero is $\bar{I}_{TT}(0)=-1.5897$

\begin{figure}[h]%[htpb] %ht, htpb (p - float, b = bottom, h=? t = top)
  %\leavevmode \includegraphics[width=1.0\textwidth]{Chap2Theory/Figures/integralsModelsDataNoEG4IttLowQ2.eps} 
  \leavevmode \includegraphics[width=1.0\textwidth]{TexmakerMyFinTh/Chap2Theory/Figures/integralsModelsDataNoEG4IttLowQ2.png} 
  \caption[Predictions for $\bar{I}^d_{TT}$]{A \chipts theoretical predictions for $\bar{I}^d_{TT}$ along with the integral calculated from the model used in the simulation for the data analysis.}
  \label{modelItt}  
\end{figure}
\end{comment}

%\subsubsection{GDH Sum Rule Measurements}
%To be added.

%\subsubsection{Dispersive Sum Rules for all \qsq}






%========================================

\begin{comment}
 \textcolor{red}{(This section still to be worked on)}
\begin{itemize}
\item \textbf{A brief history of GDH sum rule:} year and authors and a brief background to it if possible.\\ Following authors derived the same result independently (about the same time)
\begin{enumerate}
\item Gerasimov
\item Drell-Hearn
\item Hosoda and Yamamoto %Slifer p47
\end{enumerate}
\item General Principles/Assumptions that goes into the derivation:
\begin{enumerate}
\item \textbf{Lorentz and gauge invarianc}e in the form of \textbf{low energy theorem of Low, Goldman and Goldberger} %citation from K. Slifer (pg 47)
\item \textbf{Unitarity} in the form of the \textbf{optical theorem}
\item \textbf{Causality} in the form of an \textbf{unsubtracted dispersion relation for forward Compton scattering}. %citation
\end{enumerate}

 
\item \textbf{Implication of the sum rule:} \\ 
The sum rule implies that if a particle has a non-zero anomalous magnetic moment, then the particle has some structure and is not a point-like particle. This is because, to have non-zero right hand side, one must have non-zero value of the left hand side of the sum rule, which means the particle has some excited states. And, for a particle to have excited states, it must have some internal structure.

In addition to the benefit of that implication, the sum rule and its extension to $Q^2>0$ provides an important testing ground for various theoretical predictions based on QCD and its effective theories/models.
\end{itemize}
\end{comment}



\pagebreak   %12/6/13 (because nothing else that I tried worked)
\subsection{Generalized Forward Spin Polarizability $\gamma_0$}
%Read section "Generalization of the GDH approach" at pg 48 of KuhnChenLeader, also sec 3.2 (pg 49), 3.4 (pg 51???),3.5 (pg 54), para at 57 starting with "Results for gamma_0 ...", pg 61 (Summary and Outlook)
Polarizabilities are fundamental observables (quantities) that characterize the structure of composite objects such as nucleons or deuteron. They reflect the response to external perturbations such as %due to %SEK
external electromagnetic fields. Like the GDH sum, they are also %some %SEK
integrals over the excitation spectrum of the target and their derivations rely on the same basic assumptions. At the real photon point, for example, the electric and magnetic polarizabilities $\alpha$ and $\beta$ represent the target's response to %an %SEK
external electric and magnetic fields respectively. %And, the %SEK
The generalized polarizabilities represent the extensions of these quantities to the case of virtual photon Compton scattering. Because the integrals defining the polarizabilities involve weighting by some powers of $1/\nu$ or $x$, they converge faster than the first moments and thus are more easily %it is easier to be %%SEK
determined from low energy measurements. In other words, they have reduced dependence on the extrapolations to the unmeasured regions at large $\nu$, and higher sensitivity to the low energy behavior of the cross sections (particularly the threshold behavior), thus providing better testing grounds for theoretical predictions such as from \chipts and phenomenological models \cite{propE06_017}. % Deuteron proposal

%Some basic intro of polarizability in some M. Burkhard paper or in wikipedia
%We saw in the derivation section of the GDH sum rule that it 
The GDH sum rule comes from the first term of the %following 
low energy expansion %(see equation \ref{LEexp2}) 
of the forward Compton amplitude \cite{MBurkardtG2}. %pg 6 proton proposal
Likewise, we get another sum rule from the second, i.e., the next-to-leading term (which is in the third power of $\nu$). The second coefficient of the expansion is known as the forward spin polarizability $\gamma_0$ and by comparing the coefficients of the $\nu^2$ terms on both sides (coming from the dispersion relations on the left side and from the low energy expansion on the right side) gives us the following expression for the polarizability \cite{propE03_006}:
\begin{eqnarray}
\label{polzGm0}
\gamma_0 = -\frac{1}{4 \pi^2} \int_{thr}^{\infty } \frac{\sigma_{\frac{1}{2}} - \sigma_{\frac{3}{2}}}{\nu^3} d\nu
\end{eqnarray}
%As stated earlier, because of the $\nu^3$ weighting, this integral converges more rapidly in energy than the GDH integral and therefore can more easily be determined by low beam energy measurements. %The first measurement of this quantity for proton target at the real photon point was done by the GDH experiment at Mainz \cite{propE03_006}. 

%Generalization to Q2>0 (see DrechselPasquiniVanderhaegen paper starting from pg 22 and see eq 105 at pg 31)
Now, by considering the case of forward scattering of a virtual photon  % doubly virtual compton scattering (VVCS) 
and using the same general approach as for getting the generalized GDH sum rule, the $\mathcal{O}(\nu^3)$ (NLO) term in the low energy expansion of VVCS (doubly virtual Compton scattering) amplitude $g_{TT}(x,Q^2)$ gives the following generalization of the forward spin polarizability \cite{DrechselPasqVan} \cite{pLeaderKuhnChen}:

\begin{eqnarray}
\label{Gm0Gen}
\gamma_0(Q^2) \equiv \gamma_{TT}(Q^2) &=& \frac{16 \alpha M^2}{Q^6} 
\int^{x_0}_0  \left[ g_1(x,Q^2) - \frac{4 M^2 x^2}{Q^2} g_2(x,Q^2) \right] ~ x^2  dx  \\
\label{Gm0GenAf}
&=& \frac{16 \alpha M^2}{Q^6} \int^{x_0}_0  A_1(x,Q^2)F_1(x,Q^2) ~ x^2  dx 
\end{eqnarray}
where $\alpha = \frac{e^2}{4 \pi}$ is the fine structure constant. At large \qsq, the $g_2$ dependent term in the integrand becomes negligible and $\gamma_0$ reduces to the third moment of \gone \cite{DrechselPasqVan}.
%https://en.wikipedia.org/wiki/Fine-structure_constant In natural units \alpha = \frac{e^2}{4\pi} otherwise \alpha = \frac{e^2}{4\pi \epsilon_0  \hbar c}   %\hbar worked  (10/23/13: http://www-users.york.ac.uk/~pjh503/LaTeX/equations.html)

In exactly the same manner, from the $\mathcal{O}(\nu^2) $ term of the low energy expansion of the VVCS amplitude $g_{LT}(x,Q^2)$ one gets another polarizability - the generalized longitudinal-transverse polarizability as follows: % pg 33 eq 110 of  \cite{DrechselPasqVan} or eq 127 of KCL
\begin{eqnarray}
\label{Del0Gen}
\delta_0(Q^2) \equiv \delta_{LT}(Q^2) = \frac{16 \alpha M^2}{Q^6} 
\int^{x_0}_0  \left[ g_1(x,Q^2) + g_2(x,Q^2) \right] ~ x^2  dx 
\end{eqnarray}
This latter polarizability is not considered here because we did not measure the transverse target configuration.

Because the generalized polarizabilities can be expressed with the moments of the structure functions, it is possible to measure them using measurements of the structure functions. As stated earlier, because of the weighting by some powers of $\nu$ or $x$, these integrals converges more rapidly in energy than the GDH integral and therefore can more easily be determined by low beam energy measurements. %The first measurement of this quantity for proton target at the real photon point was done by the GDH experiment at Mainz \cite{propE03_006}. 
These integrals are valuable because they shed light on the long distance (soft), non-perturbative aspects of the target structure. The integrals are possible to be calculated using effective or approximate theories such as \chipts and lattice methods. Thus the measurements of these quantities provide benchmark tests of such theories.


The first measurement of $\gamma_0$ for a proton target at the real photon point was done by the GDH experiment at Mainz  \cite{propE03_006}.
Recently the JLab EG1b experiment has provided some finite \qsqs results for both deuteron (see Fig. \ref{modelGm0}) as well as nucleon targets \cite{nGuler_th}. 
\begin{comment}
Some \chipts calculations \cite{KaoPV04} \cite{BEKM13} as well as phenomenological predictions  \cite{maid07} are also available and have been used to compare with the available measurments.

%Data from measurements at small momentum transfer give information on the long range phenomena (Goldstone bosons and collective resonances), whereas the large momentum transfer data give information on primary degrees of freedom (quarks and gluons). The data extends over a wide range of phenomena from coherent to incoherent processes and from the generalized spin polarizabilities on the low energy side (confinement region with hadronic degrees of freedom) to higher twist effects in DIS side (perturbative region with asymptotically free partonic degrees of freedom). \cite{DrechselSmRlPol}
\begin{figure}[h]%[htpb] %ht, htpb (p - float, b = bottom, h=? t = top)
  %\leavevmode \includegraphics[width=1.0\textwidth]{Chap2Theory/Figures/integralsModelsDataNoEG4Gm0LowQ2.eps} 
  \leavevmode \includegraphics[width=1.0\textwidth]{TexmakerMyFinTh/Chap2Theory/Figures/integralsModelsDataNoEG4Gm0LowQ2.png} 
  \caption[Predictions for $\gamma^d_0$ and some data]{Some theoretical predictions for $\gamma^d_1$ together with the recently measured EG1b data.}
  \label{modelGm0}  
\end{figure}
	
\end{comment}
