% last modified date 5/17/12 (then on 1/30/13; 9/17/13)
\section{Pion Contamination Corrections}
%Look at the technote that I wrote 
%Our method of physics quantity extraction depends directly on the precise knowledge of the amount/counting of electrons that are scattered off the polarized deuterons in the \nd3 targets. But the CLAS detector that we use cannot distinguish whether a detected electron candidate really scattered off the intended target or came from other sources such as scattering from other target species or from secondary sources such as \epem pair production. At the very best, it can only distinguish between one type of particle from another. Even then, quite often, the detector mistakes the identity of other entirely different type of particles such as negative pions and registers them as electrons thus overestimating the amount of scattered electrons by significant margins. Therefore, it is imperative that we must correct for such overestimations of the electron counts before we proceed for further calculations and the resultant conclusions. In this section, the work done to estimate and correct for the amount of negative pion contamination in EG4 data will be discussed. 

% See Nevzat's thesis page 122 & 216 to expand this introductory section (Talk about CC threshold for  pions and electrons, and so on)
One of the two major sources of backgrounds in the measured EG4 electron rates comes from misidentified negatively charged pions (\pim) that produce similar set of signals as electrons in various detector components and thus pass the electron ID cuts. In the EG4 experiment, signals from the electromagnetic calorimeter (EC) and Cherenkov counter (CC) are used to discriminate electrons from pions, but even with stringent conditions on these signals, some of the pions get misidentified as electrons. To avoid limiting statistics too much in order to minimize the final statistical error in a given kinematic bin, a trade-off in purity versus efficiency (statistics) is made by quantifying the amount of this kind of contamination.

\subsection{Method}
%\section{Binning}
%  Work directory: /u/home/adhikari/Acceptance/BckgrndAnalysis
First, % of all, 
the whole kinematic space covered by EG4 is divided into 90 two-dimensional bins - 9 in p and 10 in %along p-axis and 10 along \th-axis
\thns\footnote{For 2 GeV or higher beam energy data sets, the p-bin boundaries are chosen as (0.30, 0.60, 0.90, 1.20, 1.50, 1.80, 2.20, 2.60, 3.00) and (0.30, 0.45, 0.60, 0.75, 0.9, 1.1, 1.4) for others. And, for \thns, the boundaries are (5.0, 6.0, 7.0, 8.0, 9.0, 10.0, 12.0, 15.0, 19.0, 25, 49). The choice of the binning was rather arbitrary. Nevertheless higher statistics region was divided into relatively finer bins (event population peaks around \thns = 10 degrees).}.

%\subsection{Event Selection}
%file $N_{phe}$NH3n.C is compiled and used with the directive PI_CONT_STUDY_$N_{phe}$ & PI_CONT_NH3/PI_CONT_ND3 enabled (same program is used for pair-symm. contam. histograms too, but with different directives).
For each kinematic bin, a histogram of the number of photo-electrons (variable `$N_{phe}$' in the data ntuple) produced by the electron candidates (selected using the standard particle selection conditions (cuts) except that     no cut on `$N_{phe}$' is included %GED '$N_{phe}$' itself) 
is made (see Fig. \ref{fignphePiC}). Likewise, using a very stringent set of cuts, a similar histogram is made for the cleanest possible sample of pion candidates in the same kinematic bin. %These selection conditions are as follows: %\item Event selection: clean pion samples (different set of strict cuts) and contaminated electron samples (standard cuts)

%\subsection{Electron Selection Cuts}
%\subsection{Pion Selection Cuts}

%Next, a 7th order polynomial is fit to the $N_{phe}$ histogram for electrons in the $N_{phe}$ range extending from $N_{phe}$ = 2.5 to $N_{phe}$=10, then this fit is extrapolated down to $N_{phe}$ = 0. The distribution represented by this fit would be what one would get if there were no contamination\footnote{Beyond $N_{phe}$ = 2.5, the electron sample is nearly pure except for a tiny fraction due to the pion tail, so any function that fits with that section of $N_{phe}$-distribution is supposed to represent truly pure electron distribution. In order to simplify the situation, we chose only from 2.5 to 10.0 rather than covering the full range beyond 10.0.}.
%\begin{comment} %11/7/13
\begin{itemize}
 
 \item {\bf Estimating the contamination in each bin:} %A polynomial fit of the impure electron sample is done beyond $N_{phe}>5$. %or so?, 
 A 7th order polynomial is fit to the $N_{phe}$ histogram for electrons in the $N_{phe}$ range extending from $N_{phe}$ = 1.8 %5.0 %2.5
  to $N_{phe}$=10.  The fit is then extrapolated down to $N_{phe}$ = 0 (see Fig. \ref{figpiCont}). Subtracting the extrapolated fit from the impure electron distribution results in the extraction of the contaminating pion peak\footnote{Beyond $N_{phe}$ = 1.8, the electron sample is nearly pure except for a tiny fraction due to the pion tail, so any function that fits that section of the $N_{phe}$-distribution is supposed to represent the pure electron distribution. In order to simplify the situation, we chose to fit only from 1.8 to 7.0 %5.0 %2.5   to 10.0  rather than covering the full range beyond 10.0.
  rather than covering the full range beyond 7.0.}. Rescaling the pure pion sample to the extracted peak gives us the distribution of the actual pion contamination over the complete range of $N_{phe}$. Finally, the counts corresponding to this rescaled pure sample in the region above the standard cut $N_{phe}>2.5$ is calculated. % (2.5)? 
 Then the ratio of this count to the impure electron count in the same standard $N_{phe}$ range gives the measured contamination for the bin.
 \item The contaminations thus evaluated for different momentum bins belonging to a particular \thns-bin are then plotted against the corresponding momenta. Then, this is fit to an exponential function.
 \item The parameters par1 and par2 of the exponential fit performed in different theta bins are next graphed together to see the presumed linear dependence.
 \item Finally, a global fit is performed on all the contaminations in different \thns- and p- bins (not on the fit parameters). The fit parameters from the earlier two fits only give us a hint to the type of the dependence, thus allowing us decide the form of the fit function.
\end{itemize}
%\end{comment}

%\textcolor{red}{Will work more here after defense, especially on the fits mentioned on the last three items above.}

%The scale sizes below are hit and trial numbers chosen to make the figures as big as possible while putting them together
\begin{figure}[H]
\centering
\subfigure[For the first in momentum and seventh in \th bin.]{
\includegraphics[scale=0.35]{TexmakerMyFinTh/FigBkGrnd/purePi_ContEl2999_0_6.png} 
\label{fignphe01}
}
\subfigure[For the first in momentum and eighth in \th bin.]{
\includegraphics[scale=0.35]{TexmakerMyFinTh/FigBkGrnd/purePi_ContEl2999_0_7.png}
\label{fignphe02}
}
\label{fignphePiC} %Effect of Dc-smear
\caption[$N_{ph}$ from pion and electron samples]{Number of photo-electrons produced in CC by clean pion and contaminated electron samples (3.0 GeV data) %\textbf{\textcolor{red}{See footnote of 'Method' subsection for more on binning.}}\\
}
\end{figure}








% http://tug.org/TUGboat/tb34-1/tb106thurnherr.pdf (12/3/13)
\begin{figure}[ht]
\centering
\subfigure[For the first bin in momentum and seventh bin in \thns.]{\includegraphics[scale=0.36]{TexmakerMyFinTh/FigBkGrnd/pi2elRatio2999_0_6.png}
\label{figpiCont306}   }
\quad
\subfigure[For the first bin in momentum and eighth bin in \thns.]{\includegraphics[scale=0.36]{TexmakerMyFinTh/FigBkGrnd/pi2elRatio2999_0_7.png}
\label{figpiCont307}  }
\subfigure[Fits in the $\theta(9.0,10.0)$ bin for 1.339 GeV data.]{\includegraphics[scale=0.36]{TexmakerMyFinTh/FigBkGrnd/GlbFitLinThEbBoth_E1ppt.png} 
\label{figpcFit1}    }
\quad
\subfigure[Fits in the $\theta(9.0,10.0)$ bin for 2.0 GeV data.]{\includegraphics[scale=0.36]{TexmakerMyFinTh/FigBkGrnd/GlbFitLinThEbBoth_E3ppt.png}
\label{figpcFit2}    }
%\caption{Main figure caption}
\label{figpiCont}
\caption[Calculation of pion contamination and fits.]{The top row plots show the calculation of pion contamination of electrons for the given kinematic bins of 3.0 GeV data. The dotted black line indicated by the label ``Raw El'' in the legends of each of the two plots are the contaminated electrons. Likewise, the line labeled ``El Fit'' is a polynomial fit to the electron distribution (in this case fitted from Nphe=1.8 to 7.0, but extrapolated down to Nphe=0). The line labeled ``Unscaled Pi' is the pure pion distribution obtained with stringent set of cuts. ``Raw El - Fit'' is the difference between the contaminated electron sample and the polynomial fit and finally ``Scaled Pi'' is the pure pion-sample but after its scaled to match with the ``Raw El - Fit'' at the pion peak position (around 1 Nphe). The bottom row plots show the fits of the contaminations as a functions of momentum ($p$) in a given $\theta$ bin. }
%\label{figpcFit} \caption[Fits of pion contamination]{Fits of pion contamination as a function of momentum in one \ths bin.}
\end{figure}







% % % % % % % % % % % % % % % % % % % % % % % % % % % % % % % % % % % % % % % % % % % % % % % % % % % % % % 
\begin{comment}



% idea source: http://texblog.wordpress.com/2007/08/28/placing-figurestables-side-by-side-subfigure/ :Placing figures/tables side-by-side (\subfigure)
% Can include any number of figures/tables, not just two.
%Remake the following two plots with improved fit and also specify the exact range of the p- and th- bins.
%The scale size below was a hit and trial number chosen to make the figures as big as possible while putting them together
\begin{figure}[h]
\centering
\subfigure[For the first bin in momentum and seventh bin in \thns.]{\includegraphics[scale=0.35]{FigBkGrnd/pi2elRatio2999_0_6.eps}
\label{figpiCont306}
}
\subfigure[For the first bin in momentum and eighth bin in \thns.]{\includegraphics[scale=0.35]{FigBkGrnd/pi2elRatio2999_0_7.eps}
\label{figpiCont307}
}
\label{figpiCont} %Effect of Dc-smear
%\caption[Optional caption for list of figures]{Caption of subfigures \subref{figsubfig1}, \subref{figsubfig2} and \subref{figsubfig3}}
\caption[Calculation of pion contamination.]{Calculation of pion contamination of electrons for the given kinematic bins of 3.0 GeV data. The dotted black line indicated by the label ``Raw El'' in the legends of each of the two plots are the contaminated electrons. Likewise, the line labeled ``El Fit'' is a polynomial fit to the electron distribution (in this case fitted from Nphe=1.8 to 7.0, but extrapolated down to Nphe=0). The line labeled ``Unscaled Pi' is the pure pion distribution obtained with stringent set of cuts. ``Raw El - Fit'' is the difference between the contaminated electron sample and the polynomial fit and finally ``Scaled Pi'' is the pure pion-sample but after its scaled to match with the ``Raw El - Fit'' at the pion peak position (around 1 Nphe). }
\end{figure}

% % % % % % % % % % % % % % % % % % % % % % % % % % % % % % % % % % % % % % % % % % % % % % % % % % % % % % 

\pagebreak %12/3/13

\begin{figure}[h]
\centering
\subfigure[Fits in the \thns(9.0,10.0) bin for 1.339 GeV data.]{
\includegraphics[scale=0.35]{FigBkGrnd/GlbFitLinThEbBoth_E1ppt.eps} 
\label{figpcFit1}
}
\subfigure[Fits in the \thns(9.0,10.0) bin for 2.0 GeV data.]{
\includegraphics[scale=0.35]{FigBkGrnd/GlbFitLinThEbBoth_E3ppt.eps}
\label{figpcFit2}
}
\label{figpcFit} %Effect of Dc-smear
\caption[Fits of pion contamination]{Fits of pion contamination as a function of momentum in one \ths bin.}
\end{figure}







\begin{figure}[ht]
\centering
\subfigure[Pion contamination]{
\includegraphics[scale=0.35]{FigBkGrnd/GlbFitLinThEbBoth_E4ppt.eps} 
\label{figpcFit3}
}
\subfigure[Pair-symmetric contamination]{
\includegraphics[scale=0.35]{FigBkGrnd/psContVsP_alFits3ppt.eps}
\label{figpscFit}
}
\label{figpcFits} %Effect of Dc-smear
\caption[Fits of pion contamination]{Fits of pion contamination.}
\end{figure}
\end{comment}


From the study, it is found that the typically pion contamination is less than 1 \%.
