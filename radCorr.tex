\pagebreak   %12/6/13 (because nothing else that I tried worked)
%It was originially in introChp4Sim.tex (made a separate file on 9/7/13)
\section{Radiative Corrections}
%\textbf{\textcolor{red}{Comment: This section may be moved to another chapter later on, leaving only a passing reference to the subject here.}}\\ %\\%\newline
%\textbf{Elaborate more on corrections.}
% Nevzat's thesis page 258 or 273
The physics quantities that we seek to extract from measurements are theoretically defined or interpreted and calculated in terms of the cross-section of the so called ``Born'' scattering process, which is represented by the simplest possible Feynman diagram i.e., by the lowest order approximation of a single photon exchange process. However, the measured cross-sections also contain contributions from 
higher order electromagnetic processes, %other radiative terms and sources, 
which must be accounted for %removed 
before extracting the quantities of our interest. These additional contributions are grouped into two categories - {\bf internal and external} radiative corrections. 

The {\bf internal corrections} are the contributions from the higher order QED processes (higher order Feynmann diagrams) which occur during the interaction. % are termed the {\bf internal} radiative corrections. 
These include the correction for the internal Bremsstrahlung (i.e., the emission of a real photon while a virtual photon is being exchanged with the target) by the incoming or the scattered electron), the vertex correction (in which a photon is exchanged between the incoming and the scattered electron), and the correction for the vacuum polarization of the exchanged virual photon (\epems loops). 

External %And, the {\bf external 
corrections include those that account for the energy loss (mainly by the Bremsstrahlung process) of electrons well before or after the interaction while passing through the target material and the detector. % field.

If the beam electron radiates a photon before the scattering, the kinematics of the actual process will be different from the the one calculated with the nominal beam energy. Likewise, if the radiation occurs after the scattering, the actual energy and momentum of the scattered electron will be different from what is calculated normally (i.e., without any radiation). The effect %is very big 
can be quite large for elastic scattering. 

% ==== R.Fersch thesis: pdf-viewer page: 235, actual page# not available
%At higher values of \qsqs, the total cross-section of the inclusive process is approximately equal to that of the Born scattering (one photon exchange process), but at lower \qsq, the approximation doesn't work and one has to consider higher order terms: % (see Fersch for more)
%Thus, the radiative effect becomes especially important for elastic scattering because
%%%%%%% The resulting energy loss may affect the kinematic calculations. The effect becomes especially important for elastic scattering because the elastic cross section grows rapidly as the beam energy decreases, which increases the probability for radiation of a high energy photon followed by elastic scattering. This creates a radiative elastic tail extending from the elastic peak into the inelastic region. The corrections depend on the experimental conditions.
